%% LyX 2.3.6 created this file.  For more info, see http://www.lyx.org/.
%% Do not edit unless you really know what you are doing.
\documentclass[english]{article}
\usepackage[T1]{fontenc}
\usepackage[latin9]{inputenc}

\makeatletter

%%%%%%%%%%%%%%%%%%%%%%%%%%%%%% LyX specific LaTeX commands.
%% Because html converters don't know tabularnewline
\providecommand{\tabularnewline}{\\}

%%%%%%%%%%%%%%%%%%%%%%%%%%%%%% User specified LaTeX commands.
\usepackage{babel}






\makeatother

\usepackage{babel}
\begin{document}
\title{\texttt{\textbf{Brick Games Manual}}}
\author{by JakeTheSillySnake (@ginzburg\_jake on Telegram)}
\date{Last updated September 2024}

\maketitle
\tableofcontents{}

\section{Introduction}

This document describes how to install, run, check and remove Snake
and Tetris on Linux and Unix-like systems. Please note that the program
was written and tested on Ubuntu 22.04 LTS, so its behaviour may differ
if you use other versions or OS distributions.

\section{Installation \& Running}

\subsection{Prerequisites}

Correct compilation and running of the program depends on other utilities
and libraries. Check that you have their latest versions before proceeding: 
\begin{description}
\item [{Basics,~Command~Line~Interface}] gcc, make 
\item [{Desktop~Interface}] Qt library, qmake
\item [{Testing}] GTest library 
\item [{GCOV~Report}] gcov, lcov 
\item [{Leaks~Check}] valgrind 
\item [{Convert~Manual~to~DVI}] texi2dvi 
\end{description}

\subsection{Setup}

Download or clone (\texttt{git clone <link\_to\_repo>}) the source
repository to where you can easily find it. Then type and run the
following commands in the terminal: 
\begin{enumerate}
\item \texttt{cd <path-to-games-folder>/src} 
\item \texttt{make install} 
\end{enumerate}
Now the program is compiled, placing all necessary files in a single
folder named \texttt{games/}. To start either game, run: 
\begin{enumerate}
\item For Command Line Interafce:\texttt{ make tetris\_cli OR make snake\_cli} 
\item For Desktop Interface:\texttt{ make tetris\_desk OR make snake\_desk} 
\end{enumerate}
If there are errors, you're likely missing some packages. Check \textbf{Prerequisites}.

\subsection{How to play}

This program is a reimagining of classic Tetris and Snake. In \textbf{Tetris},
your task is to make horizontal lines from randomly generated falling
pieces. Each completed line earns you points, insreasing the level
of difficulty and speed. You lose once the pieces reach the upper
border. Pay attention to the following controls: 
\begin{description}
\item [{%
\begin{tabular}{|c|c|c|}
\hline 
\textbf{Action} & \textbf{CLI key} & \textbf{Desktop key}\tabularnewline
\hline 
\hline 
\textbf{(Re)start} & S & X\tabularnewline
\hline 
\textbf{Pause} & Space & Space\tabularnewline
\hline 
\textbf{Quit} & Q & Exit button\tabularnewline
\hline 
\textbf{Left} & Left Arrow & A\tabularnewline
\hline 
\textbf{Right} & Right Arrow & D\tabularnewline
\hline 
\textbf{Down} & Down Arrow & S\tabularnewline
\hline 
\textbf{Rotate} & Enter & W\tabularnewline
\hline 
\end{tabular}}]~
\end{description}
In \textbf{Snake}, you control a ``snake'' that moves around a field.
Collecting ``apples'' that appear on the field earns you points,
increasing the snake's length and speed. You lose if the snake crashes
into the walls or itself. 
\begin{description}
\item [{%
\begin{tabular}{|c|c|c|}
\hline 
\textbf{Action} & \textbf{CLI key} & \textbf{Desktop key}\tabularnewline
\hline 
\hline 
\textbf{(Re)start} & S & X\tabularnewline
\hline 
\textbf{Pause} & Space & Space\tabularnewline
\hline 
\textbf{Quit} & Q & Exit button\tabularnewline
\hline 
\textbf{Left} & Left Arrow & A\tabularnewline
\hline 
\textbf{Right} & Right Arrow & D\tabularnewline
\hline 
\textbf{Down} & Down Arrow & S\tabularnewline
\hline 
\textbf{Up} & Up Arrow & W\tabularnewline
\hline 
\end{tabular}}]~
\end{description}

\section{Game structure \& testing}

The games stick to the C++17 language and standard libraries, with
the desktop interface provided by the Qt5. Both use a finite state
machine (FSM) to manage the game's state. The source code can be found
in \texttt{brick\_game/tetris} and \texttt{brick\_game/snake}. It
controls switching between the states: 
\begin{description}
\item [{Start}] beginning of the game 
\item [{Spawn}] generation of a new tetris piece/apple 
\item [{Shifting}] tetris piece shifts downs/snake moves forward after
a set time period 
\item [{Moving}] tetris piece/snake responds to the user's actions 
\item [{Colliding}] tetris piece reaches the bottom or lands on another
piece/snake collides with itself or walls 
\item [{Pause}] the game stops updating the field 
\item [{Gameover}] the game ends when tetris pieces reach the top/snake
collides with itself or walls 
\item [{Exit}] termination of the game 
\end{description}
The games close if the files storing the high scores can't be accessed.
FSM continuously returns updated game field and stats depending on
the user's action.

Additionally, Snake uses a VCM model, where the source code is divided
between the model, controller and viewer classes. The model contains
all the relevant information about the game, the controller accepts
the user actions and passes them to the model, and the viewer displays
the current model and processes the user input.

The backend libraries can be tested with GTest: 
\begin{enumerate}
\item To run tests:\texttt{ make test\_snake} \texttt{OR make test\_tetris} 
\item To display test coverage:\texttt{ make report\_snake OR make report\_tetris} 
\item To check for leaks: \texttt{make valgrind } 
\end{enumerate}
Running \texttt{make} or \texttt{make all} will reinstall and compile
the program. You can get DVI documentation with \texttt{make dvi}
or a distribution .tar.gz package with \texttt{make dist}.

\section{Deinstallation}

Simply run \texttt{make uninstall }or\texttt{ make clean. }This will
remove the \texttt{games/} directory but not the original download,
which can be safely deleted afterwards.

\subparagraph{If you wish to suggest an improvement or report a bug, contact me
@ginzburg\_jake (Telegram) or JakeTheSillySnake (GitHub).}
\end{document}
